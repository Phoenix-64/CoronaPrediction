\documentclass {article}
\usepackage[utf8]{inputenc}
\usepackage{listings}
\usepackage{hyperref}
\usepackage{verbatim}
\usepackage{graphicx}


\title {Corona number Prediciton}
\author {Pionczewski, Benjamin \and Marxer, Elina}
\date {\today {}}



\begin {document}
\maketitle
\vspace {3cm}

\tableofcontents

\clearpage

\section {Modle}

The prediction is made by fitting an exponential curve ($N_{(t)} = N_0*e^{kt}$) to a dataset.

For the fitting, the data is converted to a linear scale by taking the ln of the $y$ axis:
\[\omega_i = \ln(y_i)\]

\lstinputlisting[language=Python, firstline=11, lastline=13]{D:/Programming/CoronaPrediction/fitting.py}

And then the least square method is used to compute values $a$ and $b$ to fit the line ($\omega = a + by$) to the dataset:
\[ b = \frac {\sum(x_i -\overline{x})(\omega_i - \overline {\omega}) } {\sum(x_i - \overline{x})^2} \]
\[ a = \overline {\omega} - b\overline{x} \]
This is done in fitting.py using np.mean for the averages:
\lstinputlisting[language=Python, firstline=42, lastline=55]{D:/Programming/CoronaPrediction/fitting.py}

\newpage

To convert the linear line ($\omega = a + bx$) back to an exponential we take $e$ to the power of $a$ for $N_0$, b can be left as is and taken as k
\[N_0 = e^a \]
\[k = b\]
So we get the exponential curve:
\[ N_{(t)} = e^{a + bt} \]

\lstinputlisting[language=Python, firstline=31, lastline=34]{D:/Programming/CoronaPrediction/fitting.py}

Where $t$ is the time between the first point in our sample dataset and the date we want to predict the cases for.

\section {Data}
The data used for the prediction comes from the official BAG website\footnote{\url{https://www.bag.admin.ch/bag/en/home/krankheiten/ausbrueche-epidemien-pandemien/aktuelle-ausbrueche-epidemien/novel-cov/situation-schweiz-und-international.html}}

Then the excel sheet is converted to a .csv file and converted to a list:
\lstinputlisting[language=Python, firstline=11, lastline=14]{D:/Programming/CoronaPrediction/main.py}

For an easier source data selection, a \verb=generate_dates= function is implemented. Just give the starting date and the range.
\lstinputlisting[language=Python, firstline=45, lastline=56]{D:/Programming/CoronaPrediction/main.py}
\newpage
And the daily cases extracted. For a more accurate dataset, only the weekdays are counted.
\lstinputlisting[language=Python, firstline=24, lastline=37]{D:/Programming/CoronaPrediction/main.py}


\section {Result}
Based on the data from the 15.02.2021 to the 31.03.2021, the modle predicts approximately \emph{3400} new cases for the 27. April 2021.
On tests the modle was generally a bit above the true cases, so this is a rather pessimistic projection.
The complete graph\footnote{An interactive version can be found here: \url{http://magicpio.internet-box.ch/case_graph.html}}: shows the projection from the first data point 15.02.2021 (0) to the 27.04.2021 (70).
\begin{figure}[h]
    \centering
    \includegraphics[width=0.8\textwidth]{case_plot}
    \caption {The projection of the cases against time.}
    \label{fig:Cases}
\end{figure}


The whole modle is available be found on github \footnote{\url{https://github.com/Phoenix-64/CoronaPrediction}}
	
\end {document}